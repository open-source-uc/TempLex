% %%%%%%%%%%%%%%%%%%%%%%%%%%%%%%%%%%%%%%%%%%%%%%%%%%%%%%%%%%%%%%
% IMPORTANTE: Cambiar el compilador a XeLaTeX en las opciones %
%               Creditos Github: @diegocostares               %
%                Disponible en: @open-source-uc               %
% %%%%%%%%%%%%%%%%%%%%%%%%%%%%%%%%%%%%%%%%%%%%%%%%%%%%%%%%%%%%%%
\documentclass{style}

\begin{document}
% Renombre
\renewcommand{\tablename}{Tabla}
\renewcommand{\figurename}{Imagen}
\renewcommand*{\lstlistingname}{Código}

((* if tex.encabezado.visible *))
% %%%%%%%%%%%%%%%% ENCABEZADO %%%%%%%%%%%%%%%%%
\fancyhead[L]{ % Encabezado a la izquierda
  \begin{picture}(0,0) \put(-40,16){\includegraphics[width=15mm]{img/logo-uc-3.pdf}} \end{picture}
  \put(8,40){\textcolor{gray}{\scriptsize{\begin{tabular}{l}
    PONTIFICIA UNIVERSIDAD CATÓLICA DE CHILE\\
    \VAR{tex.encabezado.escuela}
  ((* if tex.encabezado.departamento *))
  \\\VAR{tex.encabezado.departamento}
  ((* endif *))
  \end{tabular}}}}
}
((* if tex.encabezado.derecha*))
% Imagen a la derecha
\rhead{\begin{picture}(0,0) \put(-85,12){\includegraphics[width=30mm]{img/\VAR{tex.encabezado.imagenderecha}}} \end{picture}}
((* else *))
\rhead{} % Quita el encabeazado a la derecha NO BORRAR
((* endif *))
((* endif *))

((* if tex.portada.visible *))
% %%%%%%%%%%%%%%%% PORTADA %%%%%%%%%%%%%%%%%
((* if tex.portada.includepdf *))
\includepdf[pages=-]{img/\VAR{tex.portada.pdfname}} % añade un pdf como portada
((* elif tex.portada.portada1 *))
((* if tex.portada.portada1 *))
\begin{titlepage}
\includegraphics[width=2cm]{img/logo-uc-4.pdf} % AGREGAMOS EL LOGO UC
\vspace*{-2.2cm} % Esta linea va con la anterior


\begin{tabular}{l}
\hspace{2cm} PONTIFICIA UNIVERSIDAD CATÓLICA DE CHILE\\
\hspace{2cm} \VAR{tex.portada.portada1.escuela} \\
\hspace{2cm} \VAR{tex.portada.portada1.departamento} \\
\end{tabular}

\begin{center}
\vspace{3cm}
{\scshape\Huge \textbf{\VAR{tex.portada.portada1.titulo}} \par}
\rule{80mm}{0.1mm}
%\vspace{1cm}

{\itshape\LARGE \VAR{tex.portada.portada1.numero} \par}
\vfill
{\Large \textbf{\VAR{tex.portada.portada1.grupo}} \par}
\vspace{0.5cm}
{\large \textbf{Integrantes:}\\\normalsize \VAR{tex.portada.portada1.integrantes} \par} 


\medskip
\textbf{Fecha de entrega:} \VAR{tex.portada.portada1.fecha}
\vspace{3cm}
\end{center}
\end{titlepage}
\newpage
((* endif *))
((* endif *))
((* endif *))

((* if tex.indices.visible *))
% %%%%%%%%%%%%%%%% INDICES %%%%%%%%%%%%%%%%%
\setstretch{1} % Espaciado entre lineas del indice
\vspace*{\fill}
    \renewcommand{\contentsname}{Índice de contenido}
    \tableofcontents

((* if tex.indices.tablas*))
    \renewcommand{\listtablename}{Índice de Tablas}
    \listoftables
((* endif *))

((* if tex.indices.figuras*))
    \renewcommand{\listfigurename}{Índice de figuras}
    \listoffigures
((* endif *))

((* if tex.indices.codigo*))
    \lstlistoflistings % Para hacer indice de codigo
((* endif *))
\vspace*{\fill}
((* endif *))

% %%%%%%%%%%%%%%%% ESPACIADO %%%%%%%%%%%%%%%%%
\setstretch{\VAR{tex.espaciado}}

% %%%%%%%%%%%%%%%% CONTENIDO %%%%%%%%%%%%%%%%%
\newpage
((* if tex.tutorial *))
\input{content/tutorial} % Eliminar en caso de no requerir
((* endif *))

((* if tex.bibliografia.visible *))
% %%%%%%%%%%%%%%%% BIBLIOGRAFIA %%%%%%%%%%%%%%%%%
\newpage
((* if tex.bibliografia.estilo == "manual" *))
\input{content/bibliografia}

((* elif tex.bibliografia.estilo == "bibtex" *))
\bibliographystyle{apacite}
\bibliography{mybib.bib} % IMPORTANTE: Requiere crear un archivo con ese nombre

((* endif *))
((* endif *))


\end{document}